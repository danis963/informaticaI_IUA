\documentclass[xcolor=pdftex,table,11pt]{beamer}
\usetheme{Warsaw}
\usepackage[utf8]{inputenc}
\usepackage[english]{babel}
\usepackage{amsmath}
\usepackage{amsfonts}
\usepackage{amssymb}
\usepackage{multirow}
\usepackage{siunitx}
\usepackage{listings}
\usepackage{tabulary}


\usepackage[highlightcolor=yellow]{../styles/code}
\author{Informática I - Instituto Unviersitario Areonáutico}
\title{Introducción a la programación en C}

\usepackage{booktabs}
\usepackage{longtable}
\newcommand*{\thead}[1]{\multicolumn{1}{c}{\bfseries #1}}

\usepackage{tikz}
\def\checkmark{\tikz\fill[scale=0.3](0,.35) -- (.25,0) -- (1,.7) -- (.25,.15) -- cycle;} 

%\setbeamercovered{transparent} 
%\setbeamertemplate{navigation symbols}{} 
%\logo{} 
%\institute{} 
%\date{} 
%\subject{} 
\begin{document}


\begin{frame}
\titlepage
\end{frame}

\begin{frame}
\tableofcontents
\end{frame}






\begin{frame}{Funciones definición}
\begin{block}{}
Las funciones permiten a los desarrolladores dividir un programa en módulos independientes.\\
\begin{itemize}
\item Funciones pre-empaquetadas de C: permiten realizar cálculos matemáticos, operaciones con cadenas de texto, operaciones de entrada y salida de datos, etc.
\item Definidas por el desarrollador: permiten realizar tareas particulares del algoritmo en cuestión.
\end{itemize}

\end{block}

 \begin{figure}
 \centering
\includegraphics[scale=0.4]{../img/exported/types_of_functions_cpp.jpg}
\end{figure}

\end{frame}



\begin{frame}{Funciones definición}

 \begin{figure}
 \centering
\includegraphics[scale=0.7]{../img/exported/Funciones.png}
\end{figure}

\end{frame}






\begin{frame}[allowframebreaks]{Anatomía de un función en C}
Las funciones se pueden clasificar en 4 tipos según su naturaleza:
\begin{itemize}
\item Si Reciben y si retornan datos: calcular el promedio de dos números
\item No reciben y si retornan datos: menú de opciones
\item No reciben y no retornan datos: función saludo
\item Si reciben y no retornan datos: impresión de datos


\end{itemize}

\end{frame}


\begin{frame}[allowframebreaks]{Anatomía de un función en C}

\begin{block}{Prototipo}
Consiste en una presentación de la función. En el se define que tipo de dato retorna y si lo hace, el nombre de la misma y el tipo de dato de lo/los parámetros que recibe. En caso de ser mas de un parámetros, se los separa por comas.\\
\end{block}

Ejemplos de prototipos:
\codesetstylefrombeamer
\cppfile{../../c/functions/func_prototype.c}


Notar que para indicar que una función no recibe y/o no retorna parámetros, se utiliza la palabra reservada \textbf{void}. Además, las funciones en C pueden recibir muchos valores y de distintos tipos, pero \textbf{sólo pueden retornar un único dato}.

\newpage

\begin{block}{Estructura general de una función en C}
Luego de la declarar el prototipo de la función, se procede a la especificación formal de la misma.
\end{block}





\begin{block}{La sentencia return}
Dicha sentencia fuerza la salida inmediata de la función. Es decir que las sentencias que se encuentren después de una sentencia return(); no serán ejecutadas.\\
Esta sentencia puede ser utilizada para retornar valores, siempre y cuando el tipo de retorno \textbf{no sea void}. 
\end{block}

\newpage
Ejemplo general:
\codesetstylefrombeamer
\cppfile{../../c/functions/func_structure.c}

Ejemplos particulares:


\cppfile{../../c/functions/func_prom.c}

\cppfile{../../c/functions/func_menu.c}


\cppfile{../../c/functions/saludo.c}


\end{frame}

\begin{frame}[allowframebreaks]{Llamada a funciones}
\begin{itemize}
\item  Se realiza con el nombre de la función
\item  Si la función recibe datos, estos deben ser enviados al momento de la llamada. En orden, entre paréntsis y separados por comas
\item  Si la función NO recibe datos, se deben colocar los paréntesis vacíos
\item  Si la función retorna parámetros, debemos asignar el valor de retorno a una variable
\item  Una llamada a una función es una sentencia de C. Por ello debe colocarse el ; al final de la misma

\end{itemize}



\href{https://github.com/danis963/informaticaI_IUA/blob/main/c/src/7-funcion_sumar.c}{\beamergotobutton{Ver ejemplo I completo en github}}


\end{frame}

\begin{frame}{Ámbito de variables}
\begin{block}{Variables locales}

Se declaran dentro de una función y sólo están disponibles durante su ejecución. Cuando la función termina, son destruidas.



\end{block}


\begin{block}{Variables globales}
Globales: Se declaran fuera de las funciones y existen durante todo el ciclo de vida del programa.\\

\textbf{Su uso NO es considerado una buena práctica de programación.}



\end{block}
\href{https://github.com/danis963/informaticaI_IUA/blob/main/c/src/7-variables.c}{\beamergotobutton{Ver ejemplo I completo en github}}


\end{frame}


\begin{frame}[allowframebreaks]{Mecanismo de paso de argumentos a funciones}

\begin{block}{Paso por valor}
El valor del argumento es \textbf{copiado} en el parámetro de la subrutina, por lo cual si se realizan cambios en el mismo dentro de la función, el valor original no es modificado.

\end{block}

\codesetstylefrombeamer

\cppfile{../../c/src/7-paso_por_valor.c}



\begin{block}{Paso referencia}
Se copia la \textit{dirección de memoria} del argumento como parámetro de la función. En este caso, al realizar cambios en parámetro formal este si se ve afectado.
\end{block}
\codesetstylefrombeamer
\cppfile{../../c/src/7-paso_por_referencia.c}


¿Qué significan los símbolos * y \&?


\end{frame}

\begin{frame}[allowframebreaks]{Punteros}
\begin{block}{Definición}
Los punteros son variables cuyos valores son direcciones de memoria. En general, esta dirección de memoria es la ubicación de otra variable.
\end{block}

La declaración de una variable puntero se realiza indicando el tipo de dato, seguido de un * y el nombre de la variable:


\codesetstylefrombeamer
\cppfile{../../c/functions/pointers.c}



\begin{block}{Operador ampersand (\&)}
Se lo conoce como \textit{operador de dirección}. Devuelve la dirección de memoria de un operando.
\end{block}

\begin{block}{Operador *}
Se lo conoce como operador \textit{operador de indirección o desreferencia}, devuelve el valor del objeto al que apunta su operando
\end{block}

Ejemplo:

\codesetstylefrombeamer
\cppfile{../../c/src/9-1-pointers1.c}


 \begin{figure}
 \centering
\includegraphics[scale=0.7]{../img/exported/pointers.png}
\end{figure}
\end{frame}

\begin{frame}[allowframebreaks]{Volviendo a las funciones: paso por referencia}
\codesetstylefrombeamer
\cppfile{../../c/src/7-swap.c}
\end{frame}

\begin{frame}[allowframebreaks]{Ejemplos}
 \begin{enumerate}
   
     \item Diseñar y codificar un programa que recibiendo desde la función main los catetos de un triángulo rectángulo, imprima en la función el resultado de la hipotenusa. \\
\href{https://github.com/danis963/informaticaI_IUA/blob/main/c/src/7-funcion_triangulo.c}{\beamergotobutton{Ver en github}}

\item Modificar el programa anterior para que el resultado sea impreso en la función main.\\
\href{https://github.com/danis963/informaticaI_IUA/blob/main/c/src/7-funcion_triangulo_return.c}{\beamergotobutton{Ver en github}}



\item Modificar el programa anterior para que el valor de los catetos sea pasado a la función por referencia y la impresión sea realizada en main.
\href{https://github.com/danis963/informaticaI_IUA/blob/main/c/src/7-funcion_triangulo_referencia.c}{\beamergotobutton{Ver en github}}


\item Modificar el programa anterior para que la impresión siga siendo realizada en main, pero la función debe tener el siguiente prototipo: void calculoHipotenusa(int *, int*, float *). 
\href{https://github.com/danis963/informaticaI_IUA/blob/main/c/src/7-funcion_triangulo_referenciafull.c}{\beamergotobutton{Ver en github}}


\newpage
     \item Diseñar y codificar un programa que implemente las siguientes funciones:
     \begin{itemize}
     \item Impresión de datos personales del desarrollador: debe ejecutarse al inicio del programa. Esta función no recibe ni retorna datos.
      \item Menú de opciones: debe permitirle al operador seleccionar entre las siguientes opciones:
      \begin{enumerate}
      
            \item Sumar dos números
            \item Restar dos números
            \item Imprimir mayor
            \item Imprimir menor
        
     \end{enumerate} 
	La opción seleccionada debe ser retornada a main.
	      \item Implementar las funciones del apartado anterior. Los dos números deben ser enviados desde main por valor. El resultado de la suma y resta debe ser impreso en la función. La impresión del mayor y el menor debe hacerse en main().
	      
   \end{itemize}
	      \href{https://github.com/danis963/informaticaI_IUA/blob/main/c/src/7-funcion_calculadora.c}{\beamergotobutton{Ver en github}}
	      
	      \newpage
     \item Modificar el programa anterior para que cada uno de los números sea enviado por referencia.      
     
     
\href{https://github.com/danis963/informaticaI_IUA/blob/main/c/src/7-funcion_calculadora_refc}{\beamergotobutton{Ver en github}}


   \end{enumerate}


\end{frame}

\begin{frame}[allowframebreaks]{Funciones y arreglos unidimensionales}
\begin{block}{}
A diferencia de las variables en las que podemos elegir pasarlas a una función por valor o referencia, los arreglos sólo se pasan a funciones \textbf{mediante referencia}.\\
Es decir que la función trabajará SIEMPRE con el arreglo original y no con una copia del mismo.
\end{block}


\begin{block}{¿Cómo le indicamos a la función que va a recibir un arreglo?}
Hay tres formas de señalar esto en el prototipo de la función:
\begin{itemize}
\item Como un array con tipo definido y sin dimensiones
\item Como un array con tipo y dimensiones definidas
\item Como un puntero

\end{itemize}


\end{block}
En C/C++ el nombre de un arreglo es un puntero al primer elemento del mismo.


\newpage
Ejemplo utilizando un array con tipo definido y sin dimensiones:
\codesetstylefrombeamer
\cppfile{../../c/functions/func_array_1d.c}

\href{https://github.com/danis963/informaticaI_IUA/blob/main/c/src/7-func_array_1d.c}{\beamergotobutton{Ver ejemplo completo en github}}

\newpage
Ejemplo utilizando un array con tipo y dimensiones definidas:
\codesetstylefrombeamer
\cppfile{../../c/src/7-2func_array_1d.c}

\href{https://github.com/danis963/informaticaI_IUA/blob/main/c/src/7-2func_array_1d.c}{\beamergotobutton{Ver ejemplo completo en github}}



\newpage
Ejemplo utilizando un puntero :) :
\codesetstylefrombeamer
\cppfile{../../c/src/7-3func_array_1d.c}


\newpage

\begin{block}{}
Escritura de una función genérica para trabajar con arreglos del mismo tipo, pero distinta longitud.

\end{block}


\codesetstylefrombeamer
\cppfile{../../c/src/7-3func_array_1d_tam.c}

\href{https://github.com/danis963/informaticaI_IUA/blob/main/c/src/7-3func_array_1d_tam.c}{\beamergotobutton{Ver ejemplo completo en github}}
\end{frame}



\begin{frame}[allowframebreaks]{Ejemplos}
 \begin{enumerate}
   
     \item Diseñar y codificar un programa que implemente las siguientes funciones:
     \begin{itemize}
     \item Saludo: imprimir los datos personales del desarrollador. Prototipo: void saludo(void);
     \item Menú: permitir al operador seleccionar una de las siguientes opciones:
     \begin{itemize}
     \item Cargar un vector de 10 elementos de tipo entero
     \item Imprimir el vector
     \item Imprimir el mayor elemento dentro del arreglo
     \item Imprimir el menor elemento dentro del arreglo
     \item Imprimir la media de todos los elementos del arreglo
	 \item Imprimir los elementos mayores a la media
	 \item Imprimir los elementos menores a la media
     \end{itemize}
     Prototipo: int menu(void);
   
     \item Cargar un vector de 10 elementos de tipo entero.\\
      		Prototipo: void cargar (int []);
	\item Imprimir el vector\\
     Prototipo: void imprimir (int []);

    \item Imprimir el mayor elemento dentro del arreglo. \\
     	     Prototipo: void imprimeMayor (int []);
     	     
      \item Impimir el menor elemento dentro del arreglo\\
     	     Prototipo: void imprimeMenor (int []);
	 
	    \item Imprimir en main la media de todos los elementos del arreglo.\\
    	Prototipo: float calculaMedia (int []);
    	
	 \item Imprimir los elementos mayores a la media, donde la media es recibida como parámetro desde la función main()\\
	  Prototipo: void imprimeMayoresMedia (int [],float);
	  
	  	 \item Imprimir los elementos menores a la media, donde la media es recibida como parámetro desde la función main()\\
	  Prototipo: void imprimeMenoresMedia (int [],float);
     \end{itemize}

\href{https://github.com/danis963/informaticaI_IUA/blob/main/c/src/5-break_cont_1.c}{\beamergotobutton{Ver en github}}


   \end{enumerate}
   
\end{frame}



\begin{frame}[allowframebreaks]{Funciones y arreglos bidimensionales}


\begin{block}{}
Para el caso de los arreglos bidimensionales, es necesario indicar en el prototipo de la función la cantidad de columnas que tiene arreglo. Este parámetro es utilizado por el compilador para poder determinar la posición de memoria de cada elemento y así poder operar con ellos.


\end{block}

\codesetstylefrombeamer
\cppfile{../../c/functions/func_array_2d.c}
\href{https://github.com/danis963/informaticaI_IUA/blob/main/c/functions/func_array_2d.c}{\beamergotobutton{Ver en github}}

\end{frame}



\begin{frame}[allowframebreaks]{Ejemplos}
 \begin{enumerate}
   
     \item Diseñar y codificar un programa declare en main un arreglo de 7 filas por 9 columnas y luego implemente las siguientes funciones:
      \begin{itemize}
      
   
     \item Cargar arreglo
     \item Imprimir arreglo
     \item Imprimir la traza de la matriz
	 \item Imprimir todas las columnas de una fila especifica. El valor de la fila es recibido por valor desde main.
	 \item Imprimir todas las filas de una columna especifica. El valor de la fila es recibido por referencia desde main..
	 \item Menú con las opciones citadas anteriormente
     
      \end{itemize}
\href{https://github.com/danis963/informaticaI_IUA/blob/main/c/src/7-funcion_triangulo.c}{\beamergotobutton{Ver en github}}

\newpage

     \item Una estación meteorológica toma muestras de temperatura y humedad una vez por hora durante 30 días Dicha información se almacena en dos arreglos bidimensionales declarados de la siguiente manera: \\
      \vspace{0.35cm}
     float temperatura [30][24]; \\    
     float humedad [30][24];\\  
    \vspace{0.5cm}
Diseñar y codificar un programa en C que implemente las siguientes funciones:
      \begin{itemize}
      
   
     \item Cargar arreglos: debe cargar valores aleatorios en los  arreglos. La temperatura debe variar entre $-10^o C$ y $50 ^o C$ y la humedad entre  $0\%$ y $100 \%$ 
     \item Imprimir las 24 muestras de un día. El número de día se recibe por parámetro desde la función main
     \item Imprimir la mayor temperatura de un día. El número de día se recibe por referencia desde la función main
     
          \item Imprimir la menor temperatura de un día. El número de día se recibe por referencia desde la función main
          
          \item Imprimir el día y la hora de la temperatura máxima y mínima

     
      \end{itemize}
\href{https://github.com/danis963/informaticaI_IUA/blob/main/c/src/7-funcion_triangulo.c}{\beamergotobutton{Ver en github}}


\end{enumerate}

\end{frame}
\end{document}