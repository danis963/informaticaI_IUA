\documentclass[xcolor=pdftex,table,11pt]{beamer}
\usetheme{Warsaw}
\usepackage[utf8]{inputenc}
\usepackage[english]{babel}
\usepackage{amsmath}
\usepackage{amsfonts}
\usepackage{amssymb}
\usepackage{multirow}
\usepackage{siunitx}
\usepackage{listings}
\usepackage{tabulary}


\usepackage[highlightcolor=yellow]{../styles/code}
\author{Informática I - Instituto Unviersitario Areonáutico}
\title{Introducción a la programación en C}

\usepackage{booktabs}
\usepackage{longtable}
\newcommand*{\thead}[1]{\multicolumn{1}{c}{\bfseries #1}}

\usepackage{tikz}
\def\checkmark{\tikz\fill[scale=0.3](0,.35) -- (.25,0) -- (1,.7) -- (.25,.15) -- cycle;} 

%\setbeamercovered{transparent} 
%\setbeamertemplate{navigation symbols}{} 
%\logo{} 
%\institute{} 
%\date{} 
%\subject{} 
\begin{document}


\begin{frame}
\titlepage
\end{frame}

\begin{frame}
\tableofcontents
\end{frame}






\begin{frame}{Funciones definición}
\begin{block}{}
Las funciones permiten a los desarrolladores dividir un programa en módulos independientes.\\
\begin{itemize}
\item Funciones pre-empaquetadas de C: permiten realizar cálculos matemáticos, operaciones con cadenas de texto, operaciones de entrada y salida de datos, etc.
\item Definidas por el desarrollador: permiten realizar tareas particulares del algoritmo en cuestión.
\end{itemize}

Las funciones se pueden clasificar en 4 tipos según su naturaleza:
\begin{itemize}
\item Si Reciben y si retornan datos: calcular el promedio de dos números
\item No reciben y si retornan datos: menú de opciones
\item No reciben y no retornan datos: función saludo
\item Si reciben y no retornan datos: impresión de datos


\end{itemize}
\end{block}
\end{frame}


\begin{frame}[allowframebreaks]{Anatomía de un función en C}

\begin{block}{Prototipo}
Consiste en una presentación de la función. En el se define que tipo de dato retorna y si lo hace, el nombre de la misma y el tipo de dato de lo/los parámetros que recibe. En caso de ser mas de un parámetros, se los separa por comas.\\
\end{block}

Ejemplos de prototipos:
\codesetstylefrombeamer
\cppfile{../../c/functions/func_prototype.c}


Notar que para indicar que una función no recibe y/o no retorna parámetros, se utiliza la palabra reservada \textbf{void}. Además, las funciones en C pueden recibir muchos valores y de distintos tipos, pero \textbf{sólo pueden retornar un único dato}.

\newpage

\begin{block}{Estructura general de una función en C}
Luego de la declarar el prototipo de la función, se procede a la especificación formal de la misma.
\end{block}

\begin{block}{La sentencia return}
Dicha sentencia fuerza la salida inmediata de la función. Es decir que las sentencias que se encuentren después de una sentencia return(); no serán ejecutadas.\\
Esta sentencia puede ser utilizada para retornar valores, siempre y cuando el tipo de retorno \textbf{no sea void}. 
\end{block}


Ejemplo general:
\codesetstylefrombeamer
\cppfile{../../c/functions/func_structure.c}

Ejemplos particulares:


\cppfile{../../c/functions/func_prom.c}

\cppfile{../../c/functions/func_menu.c}


\cppfile{../../c/functions/saludo.c}


\end{frame}

\begin{frame}[allowframebreaks]{Llamada a funciones}
\begin{itemize}
\item  Se realiza con el nombre de la función
\item  Si la función recibe datos, estos deben ser enviados al momento de la llamada. En orden, entre paréntsis y separados por comas
\item  Si la función NO recibe datos, se deben colocar los paréntesis vacíos
\item  Si la función retorna parámetros, debemos asignar el valor de retorno a una variable
\item  Una llamada a una función es una sentencia de C. Por ello debe colocarse el ; al final de la misma

\end{itemize}



\href{https://github.com/danis963/informaticaI_IUA/blob/main/c/src/7-funcion_sumar.c}{\beamergotobutton{Ver ejemplo I completo en github}}


\end{frame}

\begin{frame}{Ámbito de variables}
\begin{block}{Ámbito de variables}
\begin{itemize}
\item  Locales: se declaran dentro de una función y sólo están disponibles durante su ejecución. Cuando la función termina, son destruidas.
\item Globales: Se declaran fuera de las funciones y existen durante todo el ciclo de vida del programa
\end{itemize}

\href{https://github.com/danis963/informaticaI_IUA/blob/main/c/src/7-variables.c}{\beamergotobutton{Ver ejemplo I completo en github}}
\end{block}

\end{frame}



\end{document}