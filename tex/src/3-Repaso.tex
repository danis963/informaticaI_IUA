\documentclass[xcolor=pdftex,table,11pt]{beamer}
\usetheme{Warsaw}
\usepackage[utf8]{inputenc}
\usepackage[english]{babel}
\usepackage{amsmath}
\usepackage{amsfonts}
\usepackage{amssymb}
\usepackage{multirow}
\usepackage{siunitx}
\usepackage{listings}
\usepackage{tabulary}


\usepackage[highlightcolor=yellow]{../styles/code}
\author{Informática I - Instituto Unviersitario Areonáutico}
\title{Introducción a la programación en C}

\usepackage{booktabs}
\usepackage{longtable}
\newcommand*{\thead}[1]{\multicolumn{1}{c}{\bfseries #1}}

\usepackage{tikz}
\def\checkmark{\tikz\fill[scale=0.3](0,.35) -- (.25,0) -- (1,.7) -- (.25,.15) -- cycle;} 

%\setbeamercovered{transparent} 
%\setbeamertemplate{navigation symbols}{} 
%\logo{} 
%\institute{} 
%\date{} 
%\subject{} 
\begin{document}


\begin{frame}
\titlepage
\end{frame}

\begin{frame}
\tableofcontents
\end{frame}






\begin{frame}[allowframebreaks]{Repaso general}
	\begin{itemize}
	\item Algoritmos
		\begin{itemize}
		\item  ¿Qué es un algoritmo?
		\item  ¿Cómo está compuesto?
		\item  ¿Qué formas hay de representarlo?
		\item  ¿Cómo probamos un algoritmo?
		\end{itemize}
		
	  \item Estructura de un programa en C
		\begin{itemize}
		\item  ¿Cuales son las etapas de un programa en C?
		\item  ¿Cual es la entrada y la salida de cada una de ellas?
		\end{itemize}
		
	\item Tipos de datos en C
		\begin{itemize}
		\item  ¿Por qué debe indicarse el tipo de dato de una variables?
		\item  ¿Qué tipos de datos existe en C?
		\item  ¿Cual es el identificador de formato asociado a cada tipo de dato?
		\item  ¿Cómo ingreso datos a un programa en C desde el teclado?
		\item  ¿Cómo imprimo datos en la consola?
		\end{itemize}
		
		\item Operadores en C
		\begin{itemize}
		\item  ¿Cuáles son los operadores?
		\item  ¿Cuál es su precedencia? ¿Cómo lo resuelve C?
		\item  ¿Qué son y para que se utilizan los operadores de asignación? Ejemplos
		\end{itemize}
		
		\item Estructura switch
		\begin{itemize}
		\item  ¿Cuál es su diagrama de flujo?
		\item  ¿Y el pseudocódigo? 
		\item  ¿Donde aplicaría una estructura switch?
		\item ¿Cuál es su representación en diagrama de flujo y en pseudocódigo?
		\item  ¿Cuál es su sintaxis?
		
		\item  ¿Es siempre obligatoria la sentencia break? 
		\item  ¿Qué ocurre si no ponemos la sentencia break? 
		\item  ¿Qué ocurre si la variable de seleccion toma un valor no contemplado en los cases.? 
		\end{itemize}
		
		
		\item Operadores lógicos y relacionales
		\begin{itemize}
		\item  ¿Qué tipo de operaciones permite hacer un operador relacional?
		\item  ¿Cuál es el resultado?
		\item   ¿Qué tipos de operadores relacionales existen?
		\item  ¿Con qué tipo de datos trabaja un operador lógico?
		\item  ¿Cuál es el resultado de una operación lógica?
		\item  ¿Qué tipos de operadores lógicos existen? 
	\end{itemize}
	
			\item Estructura selectivas
		\begin{itemize}
		\item  ¿De qué tipo es la condición a evaluar?
		\item ¿Cuál es su representación en diagrama de flujo y en pseudocódigo?
		\item  ¿Qué ocurre si la evaluación de la condición resulta verdadera?
		\item   ¿Y falsa?
		\item  ¿Cuál es la diferencia entre la estructura selectiva simple y doble?
	\end{itemize}
	
			\item Estructura repetitiva for
		\begin{itemize}
		\item  ¿En qué casos es conveniente su utilización?
	    \item  ¿Cuál es su representación en diagrama de flujo y pseudocódigo?
		\item  ¿Cuál es su sintaxis?
	\end{itemize}

			\item Estructura repetitivas while y do-while
		\begin{itemize}
		\item  ¿En qué casos es conveniente su utilización?
	    \item  ¿Cuál es su representación en diagrama de flujo y pseudocódigo?
		\item  ¿Cuál es su sintaxis?
		\item  ¿Cuál es la diferencia con el ciclo repetitivo for?
		\item  ¿Cuál es la diferencia entre el ciclo repetitivo while y do-while?
	\end{itemize}
	
\end{itemize}
\end{frame}

\begin{frame}[allowframebreaks]{Ejercicios integradores - Primera etapa}
 \begin{enumerate}
   
    \item Diseñar y codificar un juego en C que permita al operador adivinar un número aleatorio comprendido entre 0 y 10.\\
    El operador sólo tiene tres intentos para adivinar el número. Para los casos fallidos, se debe imprimir un mensaje indicando si el número aleatorio es mayor o menor que la opción ingresada. Si luego de los tres intentos, el operador no logra adivinar, imprimir el mensaje "game over" y finalizar el programa.
\href{https://github.com/danis963/informaticaI_IUA/blob/main/c/src/5-break_cont_1.c}{\beamergotobutton{Ver en github}}    
    
        \item Diseñar y codificar un programa en C que permita el ingreso de notas de un examen parcial de un curso de 108 alumnos.\\
 Al finalizar, el programa debe imprimir la nota máxima, mínima, el promedio general del curso, porcentaje de aprobados y porcentaje de reprobados.
       \href{https://github.com/danis963/informaticaI_IUA/blob/main/c/src/5-break_cont_1.c}{\beamergotobutton{Ver en github}}


    
 \item Un estadio de fútbol cuenta con tres tipos de locaciones. Cada una de ellas tiene asociada un código, un precio y una cantidad como se muestra:\\
 
 
\bigskip 

\begin{tabular}{|c|c|c|c|c|}
\hline 
Código & Descripción & Costo & Cantidad disponible \\ 
\hline 
10 & Platea cubierta & \$100 & 1400 lugares \\ 
\hline 
20 & Platea descubierta & \$70 & 2000 lugares \\ 
\hline
30 & Popular & \$50 & 5000 lugares \\ 
\hline 
\end{tabular} 
\

\bigskip 
Diseñar y codificar un programa que permita a un operador vender entradas en este estadio de fútbol. El programa debe consultar el código de la butaca y si hay espacio disponible, proseguir con la venta. Caso contrario, se debe imprimir un mensaje de error y volver a comenzar desde el inicio. \\

\newpage
Cuando el operador ingrese el código 0, significa que la venta ha finalizado y se debe imprimir:
\begin{itemize}
\item Porcentaje de ocupación del estadio
\item Porcentaje de ubicación cada uno de los tipos de locaciones
\item Recaudación total

\end{itemize}
\href{https://github.com/danis963/informaticaI_IUA/blob/main/c/src/5-break_cont_1.c}{\beamergotobutton{Ver en github}}

\end{enumerate}




 
\end{frame}
\end{document}