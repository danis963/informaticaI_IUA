\documentclass[xcolor=pdftex,table,11pt]{beamer}
\usetheme{Warsaw}
\usepackage[utf8]{inputenc}
\usepackage[english]{babel}
\usepackage{amsmath}
\usepackage{amsfonts}
\usepackage{amssymb}
\usepackage{multirow}
\usepackage{siunitx}
\usepackage{listings}
\usepackage{tabulary}


\usepackage[highlightcolor=yellow]{../styles/code}
\author{Informática I - Instituto Unviersitario Areonáutico}
\title{Introducción a la programación en C}

\usepackage{booktabs}
\usepackage{longtable}
\newcommand*{\thead}[1]{\multicolumn{1}{c}{\bfseries #1}}

\usepackage{tikz}
\def\checkmark{\tikz\fill[scale=0.3](0,.35) -- (.25,0) -- (1,.7) -- (.25,.15) -- cycle;} 


\begin{document}

\begin{frame}[allowframebreaks]{Arreglos: preguntas conceptuales y snippets para analizar}


Preguntas generales:


\begin{itemize}
\item ¿Cuál es la ventaja de tener datos agrupados en arreglos?
\item ¿Qué propiedades comparten los datos almacenados en arreglos?
\item Cite ejemplos donde la utilización de arreglos reduzca la complejidad en la codificación de un algoritmo
\item ¿Cómo se define e inicializa un arreglo unidimensional?
\item ¿Cómo se define e inicializa un arreglo de dos dimensiones?
\item Para que se utiliza la directiva \#define
\end{itemize}
\newpage


Considerando el siguiente snippet, analice:
\codesetstylefrombeamer
\cppfile{../../c/snippets/snippet/8-1_snippet_array.c}


\begin{itemize}

\item ¿Cuál el valor de inicialización de los elementos del arreglo?
\item ¿Cuál es y que valor tiene el el primer elemento del arreglo?
\item ¿Cuál es y que valor tiene el el último elemento del arreglo?

\end{itemize}

\newpage
Analice la siguiente porción de código:

\codesetstylefrombeamer
\cppfile{../../c/snippets/snippet/8-3_snippet_array.c}


\begin{itemize}

\item ¿Existen errores o el programa funciona?
\item ¿Cuál es la relación entre el tipo de dato del arreglo y el valor de los índices?
\item ¿Qué tipo de dato representa el valor de los índices?

\end{itemize}


\newpage

Analice el siguiente snippet y responda:

\codesetstylefrombeamer
\cppfile{../../c/snippets/snippet/8-2_snippet_array.c}


\begin{itemize}

\item ¿Cuál es el valor que tiene cada elemento del arreglo luego de la línea 8?

\end{itemize}




\newpage

Analice el siguiente snippet y responda:

\codesetstylefrombeamer
\cppfile{../../c/snippets/snippet/8-2_snippet_array.c}


\begin{itemize}

\item ¿Cuál es el valor que tiene cada elemento del arreglo luego de la línea 8?

\end{itemize}


\newpage

Analice el siguiente snippet y responda:

\codesetstylefrombeamer
\cppfile{../../c/snippets/snippet/8-4_snippet_array.c}


\begin{itemize}

\item ¿Cuál es el valor que tiene cada elemento del arreglo luego de la línea 6?


\item ¿Existe algún problema en el código?
\end{itemize}

\newpage

Analice el siguiente snippet y responda:

\codesetstylefrombeamer
\cppfile{../../c/snippets/snippet/8-5_snippet_array.c}


\begin{itemize}

\item ¿Cuál es el valor que tiene cada elemento del arreglo luego de la línea 6?


\item ¿Existe algún problema en el código?

\end{itemize}

\newpage

Analice el siguiente snippet y responda:

\codesetstylefrombeamer
\cppfile{../../c/snippets/snippet/8-7_snippet_array.c}


\begin{itemize}

\item ¿Qué valor tiene la variable dato luego de la ejecución de la línea 3?


\end{itemize}


\newpage

Analice el siguiente snippet y responda:

\codesetstylefrombeamer
\cppfile{../../c/snippets/snippet/8-6_snippet_array.c}


\begin{itemize}

\item ¿Cuántas filas y cuántas columnas tiene el arreglo declarado?

\item ¿Cuál es el valor que tiene cada elemento del arreglo luego de la línea 4?


\end{itemize}

5\end{frame}

\begin{frame}[allowframebreaks]{Punteros: preguntas conceptuales y snippets a analizar}

\newpage
Considerando que la variable val está almacenada en la posición de memoria 0x0010, analice y responda:

\codesetstylefrombeamer
\cppfile{../../c/snippets/snippet/9-1_snippet_pointer.c}


\begin{itemize}

\item ¿Qué imprime cada una de estas lineas?


\end{itemize}



\newpage
Considerando que la variable val está almacenada en la posición de memoria 0x0010, analice y responda:

\codesetstylefrombeamer
\cppfile{../../c/snippets/snippet/9-2_snippet_pointer.c}


\begin{itemize}

\item ¿Qué imprime cada una de estas lineas?


\end{itemize}

\end{frame}



\begin{frame}[allowframebreaks]{Funciones: preguntas y snippets a analizar}
Preguntas generales:


\begin{itemize}
\item Enumere ventajas de la utilización de funciones
\item ¿Qué es el prototipo de una función?. Mencione cada una de las partes y cite ejemplos
\item ¿Cómo se realiza la llamada a una función que retorna datos?
\item ¿Cómo se realiza la llamada a una función que no retorna datos?\item ¿Cuál es la diferencia entre paso de datos por valor y por referencia?

\item Al momento de llamar a una función que recibe parámetros por valor, ¿cómo deben separarse?, ¿qué órden deben tener?


\end{itemize}
\newpage
Considerando el siguiente snippet:

\codesetstylefrombeamer
\cppfile{../../c/snippets/snippet/10-1_func.c}


\begin{itemize}

\item ¿La función retorna datos? ¿De qué tipo?
\item ¿Cuál es el nombre de la función? 
\item ¿La función recibe parámetros? ¿De qué forma?


\end{itemize}

\newpage
Considerando el siguiente snippet:

\codesetstylefrombeamer
\cppfile{../../c/snippets/snippet/10-2_func.c}


\begin{itemize}

\item ¿Es correcta la llamada a la función? ¿Qué le modificaría?
\item ¿Qué parámetro puede ser modificado por la función? ¿Por qué? 

\end{itemize}

\newpage
Considerando el siguiente snippet:

\codesetstylefrombeamer
\cppfile{../../c/snippets/snippet/10-4_func.c}


\begin{itemize}

\item ¿Es correcta la llamada a la función? ¿Qué le modificaría?
\item ¿Qué valor tienen las variables n1, n2 y n1f luego de la ejecución de la función?

\item ¿Qué valores retorna la función prom(int, int, float *)?

\item ¿Qué variables puede modificar la función prom(int, int, float *)?


\end{itemize}

\newpage
Considerando el siguiente snippet:

\codesetstylefrombeamer
\cppfile{../../c/snippets/snippet/10-5_func.c}


\begin{itemize}

\item Modifique la función para que almacene en la variable n1f el valor del promedio de las variables n1 y n2.

\item Modifique la función para que almacene en la variable n1f el promedio de n1 y n2. La función debe retornar valores
\end{itemize}
 


\newpage
Considerando el siguiente snippet:

\codesetstylefrombeamer
\cppfile{../../c/snippets/snippet/10-6_func.c}


\begin{itemize}

\item ¿Qué esperaría que imprima este programa?

\item ¿Cuál es la diferencia entre variables locales y globales?

\item ¿Qué tipo de variables hay implementadas en este programa?

\end{itemize}

\newpage
Considerando el siguiente snippet:

\codesetstylefrombeamer
\cppfile{../../c/snippets/snippet/10-7_func.c}


\begin{itemize}

\item ¿Qué esperaría que imprima este programa?

\item ¿Cuál es la diferencia entre variables locales y globales?

\item ¿Qué tipo de variables hay implementadas en este programa?

\end{itemize}

\newpage
Considerando el siguiente snippet:

\codesetstylefrombeamer
\cppfile{../../c/snippets/snippet/10-8_func.c}


\begin{itemize}

\item ¿Qué esperaría que imprima este programa?

\item ¿Cuál es valor de n1, n2 y n3 luego de la ejecución de la línea 7?

\end{itemize}


\end{frame}


\end{document}