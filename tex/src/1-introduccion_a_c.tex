
\documentclass[xcolor=pdftex,table,11pt]{beamer}
\usetheme{Warsaw}
\usepackage[utf8]{inputenc}
\usepackage[english]{babel}
\usepackage{amsmath}
\usepackage{amsfonts}
\usepackage{amssymb}
\usepackage{multirow}
\usepackage{siunitx}
\usepackage{listings}
\usepackage{tabulary}
\usepackage[highlightcolor=yellow]{../styles/code}
\author{Informática I - Instituto Unviersitario Areonáutico}
\title{Introducción a la programación en C}

\usepackage{booktabs}
\usepackage{longtable}
\newcommand*{\thead}[1]{\multicolumn{1}{c}{\bfseries #1}}


%\setbeamercovered{transparent} 
%\setbeamertemplate{navigation symbols}{} 
%\logo{} 
%\institute{} 
%\date{} 
%\subject{} 
\begin{document}


\begin{frame}
\titlepage
\end{frame}

\begin{frame}
\tableofcontents
\end{frame}



\begin{frame}{Etapas de un programa en C }

 \begin{enumerate}
   
     	\item<1->  Escritura del código fuente: se pueden utilizar diferentes editores de texto: gedit, VIM, Zinjal, Atom, Emacs, etc. 
     	
\href{https://github.com/danis963/informaticaI_IUA/blob/main/c/c_detras_de_escena/volumen_esfera.c}{\beamergotobutton{Ver en github}}


        \item<2->  Pre-procesamiento: procesa directivas como $\#$include, $\#$define e $\#$if, remueve los comentarios, etc. En general, luego del preprocesamiento, el archivo resultante contiene una gran cantidad de líneas de código.
       
		\href{https://github.com/danis963/informaticaI_IUA/blob/main/c/c_detras_de_escena/volumen_esfera.i}{\beamergotobutton{Ver en github}}
		
		
	\item<3->  Compilado: el compilador de C traduce el código del apartado anterior a assembler. 
     
		\href{https://github.com/danis963/informaticaI_IUA/blob/main/c/c_detras_de_escena/volumen_esfera.s}{\beamergotobutton{Ver en github}}
		
		
				\item<4->  Ensamblado: el ensamblado transforma el programa escrito en lenguaje ensamblador a código objeto. 
		
		\href{https://github.com/danis963/informaticaI_IUA/blob/main/c/c_detras_de_escena/volumen_esfera.o}{\beamergotobutton{Ver en github}}

		

				\item<5->  Enlazado: reune uno o más módulos en código objeto con el código existente en las bibliotecas.
		
		\href{https://github.com/danis963/informaticaI_IUA/blob/main/c/c_detras_de_escena/a.out}{\beamergotobutton{Ver en github}}

   \end{enumerate}
   
   

\end{frame}




\begin{frame}{Buenas practicas de programación}
\begin{itemize}

\item<1-> Proyectos con aproximadamente 300.000 lineas de código o menos
\begin{itemize}
\item<1->  Aplicación promedio para smartphone
\item<2->  Photoshop v1.0 (1990)
\item<3->  Quake 3 (1999)
\item<4->  Transbordador espacial STS (1981)
\end{itemize}

\item<5-> Proyectos con aproximadamente menos de 10 millones de líneas de código
\begin{itemize}
\item<6->  Age of empires - Versión online
\item<7->  Linux kernel 2.2.0
\item<8->  Windows 3.1 (1992)
\end{itemize}


\item<9-> Proyectos con aproximadamente más de 50 millones
\begin{itemize}
\item<10->  Age of empires - Versión online
\item<11->  Linux kernel 2.2.0
\item<12->  Windows 3.1 (1992)
\end{itemize}



\item<13-> Proyectos con aproximadamente más de 2 mil millones de líneas de código
\begin{itemize}
\item<14->  Google Internet Services
\end{itemize}


\end{itemize}



\end{frame}


\begin{frame} {Tipos de datos en C}
El lenguaje de programación C es \textbf{fuertemente tipado}, es decir que cada vez que se necesite declarar u operar con una variable, se debe definir y tener presente el tipo de la misma. \\
Nota: ver los tipos de datos \textit{Unsigned}
\begin{table}
\begin{tabular}{l | c | c | c | l }
Tipo de dato & Descripción & Rango  \\
\hline \hline
short & Valor entero de 2 bytes & $-2^{16}$ a $2^{16} -1 $\\ 
int & 	Valor entero de 4 bytes & $-2^{32}$ a $2^{32} -1 $\\ 
long & 	Valor entero de 8 bytes & $-2^{64}$ a $2^{64} -1 $\\ 
char & Caracteres ASCII & $-128 $ a $127$\\ 
float & Valor decimal de 4 bytes & $\num{3.4e-38} $ a $\num{3.4e-38}$\\ 
double & Valor decimal de 8 bytes & $\num{1.7e-308} $ a $\num{1.7e-308}$\\ 
bool & Valor binario &True o False\\ 
void & Tipo de dato nulo &\\ 
 string & Cadena de char  &\\ 
\end{tabular}
\caption{Tipos de datos en C}

\end{table}

\end{frame}


\begin{frame}[allowframebreaks] {Entrada y salida de datos}
Para imprimir por pantalla o ingresar datos a un programa en C, se debe informar \textbf{expresamente} el tipo de dato que se espera imprimir y/o recibir.
Esto se realiza mediante el uso de \textbf{especificadores de formato}.

\begin{table}
\begin{tabular}{l | c | l }
Tipo de dato & Especificador de formato \\
\hline \hline
short & $\%hd$ \\ 
int & 	$\%d$ \\ 
long & 	$\%li$ \\ 
char & $\%c$\\ 
float & $\%f$ \\ 
double & $\%lf$\\ 
\end{tabular}
\caption{Especificadores de formato en C}
\end{table}
 \begin{block}{Scanf}
Es una función de la librería de entrada/salida de C que permite tomar información desde el teclado.\\ 

Sintaxis: scanf("especificador de formato", $\&$variable); \\ 
Ejemplo	: scanf("$\%d$", $\&$edad); \\ 
    \end{block}

 \begin{block}{Printf}
Es una función de la librería de entrada/salida de C que permite imprimir información por pantalla.\\ 
Sintaxis: printf("especificador de formato", variable); \\ 
Ejemplo: printf("Su edad es: " $\%d$ , edad);
    \end{block}
    
Nota: para estos ejemplos se supone que la variable se ha declarado de tipo int. Ver ejemplos siguientes.
\codesetstylefrombeamer
\cppfile{../../c/src/1-0_data_types_edad_peso.c}
\end{frame}





\begin{frame}[allowframebreaks] {Operadores en C}

Operadores de asignación:

\begin{table}
\begin{tabular}{p{15mm} | p{35mm} | p{22mm} | p{22mm} | p{22mm} }
Operador & Acción & Ejemplo & Resultado\\
\hline \hline 
= & 	Asignación básica  		  & $x=10$ 	 	& x vale 10\\
*= 	& 	Asignación producto       & $x*=10$ 	& x vale 100 \\
/= 	& 	Asignación división       & $x/=2$      & x vale 50 \\
+= 		& Asignación suma         & $x+=5$  	& x vale 55\\
-= 			& 	Asignación resta  & $x-=7$  	& x vale 48\\ 

\end{tabular}
\caption{Operadores de asignación}
\end{table}



\newpage
Operadores aritméticos:

\begin{table}
\begin{tabular}{p{15mm} | p{35mm} | p{22mm} | p{22mm} | p{22mm} }
Operador & Acción & Ejemplo & Resultado\\
\hline \hline  
- 	& 	Resta & $x=12 - 3 $ & x vale 9\\
+ 	& 	Suma & $x=12 + 3 $ & x vale 15\\
* 	& 	Multiplicación &  $x=12 * 3 $ & x vale 36\\
/ 	& 	División &  $x=12 / 3 $ & x vale 4\\
--	& 	Decremento &  $x=12; x-- $ & x vale 11\\
++	& 	Incremento &  $x=12; x++ $ & x vale 13\\
\%	& 	Modulo &  $x=13 \% 2 $ & x vale 1\\
\end{tabular}
\caption{Operadores de asignación}
\end{table}



\end{frame}


\begin{frame}{Precedencia de operadores}
El lenguaje C evalúa las expresiones aritméticas en una secuencia precisa, por lo general son las mismas que aplicaríamos en el álgebra:\\

\begin{enumerate}
\item<1->  Las operaciones de multiplicación, división y módulo se resuelven primero. Si en una misma operación aparecen varias de ellas, se resuelven de izquierda a derecha

\item<2->  Las operaciones de suma y resta se aplican después. Si hubiese varias de estas, C separará en términos al igual que se haría en el álgebra

\item<3-> Luego de resueltas todas las operaciones, se procede a la asignación

\end{enumerate}
\end{frame}

\begin{frame}{Precedencia de operadores: ejemplo}


 \begin{block}{Ecuación de una recta en forma algebraica}
\begin{equation}
y(x) = a x + b
\end{equation}
    \end{block}
    

 \begin{block}{Ecuación de una recta en C}
\begin{equation}
y = a * x + b
\end{equation}


  \end{block}

 \begin{block}{Precedencia de operadores}
 \begin{enumerate}
\item<1->  Operación $a * x$
\item<2->  Operación $+b$
\item<3->  Asignación del resultado a la variable $y$
\end{enumerate}
  \end{block}
\end{frame}
\begin{frame}{Precedencia de operadores: ejemplo en C}
\codesetstylefrombeamer
\cppfile{../../c/src/1-1_presedencia_operadores.c}
\end{frame}

\begin{frame}{Estrucutra selectiva múltiple: Switch}

 \begin{block}{Aplicación}
 Permite que un programa en C tome diferentes caminos en función del valor que tome una determinada instrucción.
 
 
 \end{block}

 \begin{block}{Pseudocódigo}

    \begin{itemize}
   \item[]Según sea (variable de control)
   \begin{itemize}

     	\item[] Caso 1:
     	    \begin{itemize}
     			\item[] Instrucciones caso 1
     			\item[] Instrucciones caso 1
     			\item[] frenar
     		   \end{itemize}

        \item[] Caso 2:
     	    \begin{itemize}
     			\item[] Instrucciones caso 2
     			\item[] Instrucciones caso 2

     			\item[] frenar
 		
   			\end{itemize}
    
    \item[] Caso por descarte:
     	    \begin{itemize}
     			\item[] Instrucciones caso por descarte
     			\item[] frenar
 		
   			\end{itemize}
 		
   \end{itemize}

	\end{itemize}


 \end{block}





\end{frame}



\begin{frame}{Estrucutra selectiva múltiple: Switch}

\begin{figure}

\includegraphics[scale=0.201]{../img/exported/switch.png}
\caption{Diagrama de flujo para la estructura switch}
\end{figure}
\end{frame}


\begin{frame}{Estrucutra selectiva múltiple: Switch}
\codesetstylefrombeamer
\cppfile{../../c/src/1-2_switch_case.c}
\end{frame}





\end{document}